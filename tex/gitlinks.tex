\newcommand{\codelink}[1]{\href{https://github.com/rodluger/earthshine/blob/f3636093e89e6679d206f77898ccf54799c8cebe/notebooks/#1.ipynb}{\codeicon}\,\,}
