\documentclass[modern]{aastex62}

% Load the corTeX style definitions
\input{cortex}

\usepackage{xcolor}
\usepackage{xspace}
\newcommand{\TESS}{\emph{TESS}\xspace}
\newcommand{\todo}[1]{\textcolor{red}{#1}}

% Bibliography stuff
\bibliographystyle{aasjournal}

% Begin!
\begin{document}

% Title
\title{\TESS Photometric Mapping of a Terrestrial Planet in the Habitable Zone: 
       Detection of Clouds, Oceans, and Continents}
%\title{Detection of Continents on a Habitable-Zone Terrestrial Planet with \TESS}

% Author list
\author[0000-0002-0296-3826]{Rodrigo Luger}
\email{rluger@flatironinstitute.org}
\affil{Center~for~Computational~Astrophysics, Flatiron~Institute, New~York, NY}
%
\author[0000-0002-9328-5652]{Megan Bedell}
\affil{Center~for~Computational~Astrophysics, Flatiron~Institute, New~York, NY}
%
\author{Roland K. Vanderspek}
\affil{Kavli Institute for Astrophysics and Space Research, Massachusetts 
       Institute of Technology, Cambridge, MA}
%
\author{Christopher J. Burke}
\affil{Kavli Institute for Astrophysics and Space Research, Massachusetts 
       Institute of Technology, Cambridge, MA}

\begin{abstract}
\todo{[maybe change first sentences to focus on planet mapping \& give the game 
away less?]} The Transiting Exoplanet Survey Satellite (\TESS) mission is a 
targeted effort to detect planets smaller than Neptune around bright, nearby stars. 
While \TESS is already enjoying great success with the discovery of many new worlds, 
the strongest signal in its data is typically ignored, as it lurks in the background 
of every camera pixel. In this work, we extract this signal and demonstrate that 
it is consistent with a terrestrial planet with a rotation period of 1 day. 
Using a spherical harmonic-based reflection model developed as an extension of 
the \starry package, we are able to reconstruct the surface features of this rocky 
world. We recover a time-variable albedo map of the planet including persistent 
regions which we interpret as continental features and cloud banks. 
We argue that this planet represents the most promising detection of a habitable 
world to date, although the potential intelligence of any life on it is yet to 
be determined.
\end{abstract}

\keywords{methods: data analysis, techniques: photometric, planets and satellites: 
          oceans, planets and satellites: surfaces, planets and satellites: 
          terrestrial planets}

% Introduction
\section{Introduction}
\label{sec:intro}

%Resolving the features of planetary atmospheres and surfaces with precise 
%photometry is a crucial 
Precise photometry obtained over long periods of time can reveal the atmospheric 
and surface features of an exoplanet through the method known as phase mapping. 
\todo{(literature background \& citations)}

A new source of precise photometry is the Transiting Exoplanet Survey Satellite 
\citep[\TESS; ][]{Ricker2015}. 
Launched in April 2018, \TESS has... \todo{(summary of \TESS's capabilities, 
citations to planet discoveries etc)}

One intriguing feature of the \TESS photometric data products is a highly 
time-variable signal represented in the diffuse background across virtually 
all camera pixels. 
\todo{(talk abt Earthshine without naming it)} 
The signal has been widely attributed to a planet which we will call Sol d.

Previous attempts to map this planet from the exoplanet community have suggested 
the presence of localized surface features, but their resolving power has been 
severely limited by the duration of observations \citep{Cowan2009}. 
\TESS offers precise 2-minute cadence data spanning a wide range of illumination 
phases, enabling a level of spatial and temporal resolving power that is 
unprecedented in the exoplanetary phase mapping literature. 

Moreover, recent advances in fast analytic computation of spherical harmonics 
have made the reconstruction of such detailed maps more feasible than ever. 
\todo{(STARRY)}

In this work, we build on this methodological foundation to perform 
reflected-light mapping of planetary surface features on Sol d. 
We give an overview of the \TESS data used in Section \ref{sec:data}. 
In Section \ref{sec:methods}, we describe the methods employed to infer a map 
from these data, including the adaptation of the \starry algorithm to reflection 
mapping, the modeling of spacecraft-related systematics, and the likelihood and 
priors used. 
We present results in Section \ref{sec:results} and make comparison of these 
findings to state-of-the-art terrestrial planet maps from the literature in 
Section \ref{sec:discussion}. 
Finally, we conclude in Section \ref{sec:conclusion} with a look at future 
prospects for phase-curve mapping of terrestrial exoplanets.

\section{Data}
\label{sec:data}

\todo{(background extraction from 2-minute cadence targets)}

\todo{(outlier rejection method, trim down to x number of high-SNR lightcurves 
for computational tractability)}

\todo{(figure showing lightcurves for various targets)}

\section{Methods}
\label{sec:methods}

\begin{figure}[ht!]
    \begin{centering}
    \includegraphics[width=\linewidth]{figures/phases.pdf}
    \caption{\label{fig:phases}
             \tess view of Sol d on every date in Sectors 1
             and 2 that was included in the regression. Each image is
             labeled with the corresponding timestamp in TJD. 
             \codelink{EarthView}
             }
    \end{centering}
\end{figure}

\begin{figure}[ht!]
    \begin{centering}
    \includegraphics[width=\linewidth]{figures/data.pdf}
    \caption{\label{fig:data}
             Normalized background flux in each of ten postage stamps
             for each of Sectors 1 and 2. Cadences where Sol d was
             below or very close to the edge of the sunshade were masked.
             While the planet is in view for the majority of both orbits
             in Sector 1, it is only visible in the data at the very end
             of each orbit in Sector 2.
             \codelink{earthshine_S1_S2}
             }
    \end{centering}
\end{figure}

\subsection{Systematics Modeling}

While the astrophysical signal that we wish to analyze is present in all \todo{x} 
extracted lightcurves, its strength is modulated by time-variable systematics 
that are correlated but not identical across all sources. 
\todo{The observed lightcurves can be represented as... (systematics x 
Earthshine equation)}

We choose to model this systematic component using Pixel Level Decorrelation 
(PLD). \todo{(details, math, and a figure of the bases (possibly two-panel with 
comparison to telescope housekeeping variables?)}

\subsection{Reflected-Light Mapping with \starry}

\begin{figure}[ht!]
    \begin{centering}
    \includegraphics[width=0.5\linewidth]{figures/geometry.pdf}
    \caption{\label{fig:geometry}
             Geometry of the \starry model. In this frame, the $y$ axis points up, 
             the $x$ axis points to the right, and the $z$ axis points out of the page.
             The planet has unit radius and sits at the origin, while
             the illumination source is along the $y-z$ plane, with the sub-stellar point is
             marked by a dot. The semi-major axis of the terminator is unity, and
             the semi-minor axis of the terminator is denoted $b$. The night side
             is colored black.
             \codelink{Geometry}
             }
    \end{centering}
\end{figure}

\begin{figure}[ht!]
    \begin{centering}
    \includegraphics[width=\linewidth]{figures/starry_model.pdf}
    \caption{\label{fig:starry_model}
             Normalized surface reflectance versus time for the maximum
             likelihood \starry model in each of the four orbits of \tess
             in Sectors 1 and 2.
             \codelink{earthshine_S1_S2}
             }
    \end{centering}
\end{figure}

\todo{(brief recap of \starry)}

In \cite{Luger2019} we described how to transform between spherical
harmonic coefficients and coefficients in the polynomial basis $\tilde{p}$, whose
terms are of the form $x^i y^j z^k$ for (non-negative) integer $i$ and $j$ 
and $k = 0$ or $1$. Given a representation of a surface map in terms of
a set of coefficients in $\tilde{p}$, we showed how to compute the visible
flux by integrating each of the terms in $\tilde{p}$ across the visible
surface of the projected disk. For unocculted spheres observed in emitted
light, the area of integration was the full disk, which made the integration
trivial: the flux contribution from each term in $\tilde{p}$ was a constant
easily computed via recursion relations involving factorials.

In the present work, we are interested in \emph{reflected} light, so we must
account for the illumination by the star, which varies smoothly
across the dayside of the planet but whose derivative discontinuously changes at the 
terminator, beyond which the illumination is zero everywhere. This requires
two modifications to the \starry algorithm: we must (1) weight all integrands
by the illumination function on the dayside, and (2) modify the integration
limits to truncate the integrals at the day/night terminator.

Let us consider a right-handed coordinate frame in which the planet has unit 
radius and is located at the origin, the illumination source is along the $y-z$
plane, and the sub-stellar point is on the $+y$ axis at $x = 0$
(see Figure~\ref{fig:geometry}). In this frame, the terminator is a segment of
an ellipse, with semi-major axis $a=1$ aligned with the $x$ axis and
semi-minor axis $b$. It is straightforward to show that the sub-stellar point
is located at $y=\sqrt{1 - b^2}$ and the illumination is given by
%
\begin{align}
    I(b; x, y) &= 
    \begin{dcases}
        \sqrt{1 - b^2}y - bz & 
            \quad\quad\quad\quad\quad\quad\quad\quad\quad\quad 
            y \geq -b \sqrt{1 - x^2}
        \\
        0 & 
            \quad\quad\quad\quad\quad\quad\quad\quad\quad\quad 
            \mathrm{otherwise}
    \end{dcases}
    \quad ,
    \label{eq:illum}
\end{align}
%
where $z = \sqrt{1 - x^2 - y^2}$.
%
Since the illumination function is just a polynomial in $x$, $y$, and $z$,
weighting the terms in $\tilde{p}$ by this function keeps all terms within the
polynomial basis.
Provided we always rotate the problem such that the
sub-stellar point lies along the $+y$ axis as above, the limits of integration
are $-1 \leq x \leq 1$ and $b\sqrt{1 - x^2} \leq y \leq \sqrt{1 - x^2}$.
Computing the total reflected flux for any surface map is therefore a matter
of solving integrals of the form
%
\begin{align}
    J_{ijk}(b) &= \int_{-1}^{1} \int_{b\sqrt{1 - x^2}}^{\sqrt{1 - x^2}} x^i y^j z^k \mathrm{d} y \ \mathrm{d} x
    \nonumber\\
    &=
    \frac{
        \Gamma\left(\frac{i + 1}{2}\right) \Gamma\left(\frac{j + 1}{2}\right)
    }{
        2 \Gamma\left(\frac{i + j + k + 4}{2}\right)
    }
    \begin{dcases}
        0
        &
        \quad\quad\quad\quad\quad\quad\quad\quad\quad\quad 
        i \, \mathrm{odd}
        %
        \\
        %
            1 - b^{j + 1}
        & 
        \quad\quad\quad\quad\quad\quad\quad\quad\quad\quad 
        k = 0
        %
        \\
        %
        \frac{\sqrt{\pi}}{2} - b^{j + 1}\Gamma\left(\frac{j + 4}{2}\right)
        {_2\tilde{F}_1} \left( -\frac{1}{2}, \frac{j + 1}{2}; \frac{j + 3}{2}; b^2\right)
        & 
        \quad\quad\quad\quad\quad\quad\quad\quad\quad\quad 
        k = 1
    \end{dcases}
    \quad ,
    \label{eq:integral}
\end{align}
%
where ${_2\tilde{F}_1}$ is the regularized hypergeometric function. For all values
of $i$, $j$, and $k$ in $\tilde{p}$, ${_2\tilde{F}_1}$ reduces to simple
trigonometric functions involving $b$. Moreover, it is straightforward to
derive recursion relations for $J_{ijk}(b)$, making its evaluation extremely fast.
The algorithm for computing $J_{ijk}(b)$ is implemented in the development version
of \starry and will be discussed further in upcoming work.

\subsection{Likelihood Function \& Inference}

\todo{(priors, regularization, equation for the likelihood)}

\todo{(details of TensorFlow optimization)}

\section{Results}
\label{sec:results}

\begin{figure}[t!]
    \begin{centering}
    \includegraphics[width=\linewidth]{figures/model.pdf}
    \caption{\label{fig:model}
             \emph{Top:} Median flux across all postage stamps (black) and
             the median maximum likelihood model (blue) across Sectors 1 and 2.
             \emph{Bottom:} Median residuals about the maximum likelihood fit.
             \codelink{earthshine_S1_S2}
             }
    \end{centering}
\end{figure}

\todo{(present results with \& without time variability)}

\begin{figure}[t!]
    \begin{centering}
    \includegraphics[width=\linewidth]{figures/map.pdf}
    \caption{\label{fig:map}
             \emph{Top:} The maximum likelihood model for the surface albedo
             of Sol d, weighted by the visibility across Sectors 1 and 2. The
             dashed white lines indicate latitudes above/below which the
             visibility drops below 50\%. Black contours indicate coastlines
             derived from current state-of-the-art models of the surface geography 
             of Sol d.
             \emph{Bottom:} Actual corrected surface reflectance in the red band of the
             \emph{VIIRS} imager, averaged over the same date range and weighted
             by the same visibility function. This bandpass covers only about one
             quarter of the \tess bandpass, so deviations between the two images
             are to be expected.
             \codelink{earthshine_S1_S2}
             }
    \end{centering}
\end{figure}

The bottom panel of Figure~\ref{fig:map} shows the corrected surface
reflectance obtained from the
Visible Infrared Imaging Radiometer Suite (\emph{VIIRS}) instrument aboard the 
Suomi National Polar-orbiting Partnership (\emph{S-NPP}) satellite
\footnote{Data obtained via 
\href{https://github.com/rodluger/earthshine/blob/master/tex/figures/viirs.sh}{this wget script}.}. 
The image shows the corrected surface reflectance in the I1 band of the instrument,
averaged over the 28 days of data analyzed in this work, smoothed with a Gaussian
filter, and weighted by the \tess visibility as in the top panel.
The I1 band extends from 0.6 to 0.68 $\mu\mathrm{m}$
\footnote{\url{http://rammb.cira.colostate.edu/projects/npp/VIIRS_bands_and_bandwidths.pdf}},
and therefore covers only about one-fourth of the bandpass of \tess
(nominally 0.6 - 1.0 $\mu\mathrm{m}$).


\section{Discussion}
\label{sec:discussion}

\todo{(comparison of inferred features to ``model'' (Earth))}

\section{Conclusion}
\label{sec:conclusion}

\todo{(summarize)}

\todo{(utility for \TESS background removal)}

\todo{(talk abt prospects for doing this analysis with exoplanets)}


\acknowledgements{\todo{(Artist acknowledgement)} The authors gratefully 
acknowledge Dan Foreman-Mackey, David W Hogg, Ben Pope, \todo{and ...} 
for useful conversations. Crucial parts of this work were carried out at the 
TESS Data Workshop, hosted by Space Telescope Science Institute, and the 
Building Early Science with TESS Workshop, hosted by the University of Chicago.}
\facility{TESS}
\software{Astropy, Matplotlib, Numpy, \starry \citep{Luger2019}, TensorFlow}
% Bibliography
\pagebreak
\bibliography{bib}

\end{document}
