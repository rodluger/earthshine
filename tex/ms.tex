\documentclass[modern]{aastex62}

% Load the corTeX style definitions
\input{cortex}

\usepackage{xcolor}
\usepackage{xspace}
\newcommand{\TESS}{\emph{TESS}\xspace}
\newcommand{\todo}[1]{\textcolor{red}{#1}}

% Bibliography stuff
\bibliographystyle{aasjournal}

% Begin!
\begin{document}

% Title
\title{\TESS Photometric Mapping of a Terrestrial Planet in the Habitable Zone: 
       Detection of Clouds, Oceans, and Continents}
%\title{Detection of Continents on a Habitable-Zone Terrestrial Planet with \TESS}

% Author list
\author[0000-0002-0296-3826]{Rodrigo Luger}
\email{rluger@flatironinstitute.org}
\affil{Center~for~Computational~Astrophysics, Flatiron~Institute, New~York, NY}
%
\author[0000-0002-9328-5652]{Megan Bedell}
\affil{Center~for~Computational~Astrophysics, Flatiron~Institute, New~York, NY}
%
\author{Roland K. Vanderspek}
\affil{Kavli Institute for Astrophysics and Space Research, Massachusetts 
       Institute of Technology, Cambridge, MA}
%
\author{Christopher J. Burke}
\affil{Kavli Institute for Astrophysics and Space Research, Massachusetts 
       Institute of Technology, Cambridge, MA}

\begin{abstract}
\todo{[maybe change first sentences to focus on planet mapping \& give the game 
away less?]} The Transiting Exoplanet Survey Satellite (\TESS) mission is a 
targeted effort to detect planets smaller than Neptune around bright, nearby stars. 
While \TESS is already enjoying great success with the discovery of many new worlds, 
the strongest signal in its data is typically ignored, as it lurks in the background 
of every camera pixel. In this work, we extract this signal and demonstrate that 
it is consistent with a terrestrial planet with a rotation period of 1 day. 
Using a spherical harmonic-based reflection model developed as an extension of 
the \starry package, we are able to reconstruct the surface features of this rocky 
world. We recover a time-variable albedo map of the planet including persistent 
regions which we interpret as continental features and cloud banks. 
We argue that this planet represents the most promising detection of a habitable 
world to date, although the potential intelligence of any life on it is yet to 
be determined.
\end{abstract}

\keywords{methods: data analysis, techniques: photometric, planets and satellites: 
          oceans, planets and satellites: surfaces, planets and satellites: 
          terrestrial planets}

% Introduction
\section{Introduction}
\label{sec:intro}

%Resolving the features of planetary atmospheres and surfaces with precise 
%photometry is a crucial 
Precise photometry obtained over long periods of time can reveal the atmospheric 
and surface features of an exoplanet through the method known as phase mapping. 
\todo{(literature background \& citations)}

A new source of precise photometry is the Transiting Exoplanet Survey Satellite 
\citep[\TESS; ][]{Ricker2015}. 
Launched in April 2018, \TESS has... \todo{(summary of \TESS's capabilities, 
citations to planet discoveries etc)}

One intriguing feature of the \TESS photometric data products is a highly 
time-variable signal represented in the diffuse background across virtually 
all camera pixels. 
\todo{(talk abt Earthshine without naming it)} 
The signal has been widely attributed to a planet which we will call Sol d.

Previous attempts to map this planet from the exoplanet community have suggested 
the presence of localized surface features, but their resolving power has been 
severely limited by the duration of observations \citep{Cowan2009}. 
\TESS offers precise 2-minute cadence data spanning a wide range of illumination 
phases, enabling a level of spatial and temporal resolving power that is 
unprecedented in the exoplanetary phase mapping literature. 

Moreover, recent advances in fast analytic computation of spherical harmonics 
have made the reconstruction of such detailed maps more feasible than ever. 
\todo{(STARRY)}

In this work, we build on this methodological foundation to perform 
reflected-light mapping of planetary surface features on Sol d. 
We give an overview of the \TESS data used in Section \ref{sec:data}. 
In Section \ref{sec:methods}, we describe the methods employed to infer a map 
from these data, including the adaptation of the \starry algorithm to reflection 
mapping, the modeling of spacecraft-related systematics, and the likelihood and 
priors used. 
We present results in Section \ref{sec:results} and make comparison of these 
findings to state-of-the-art terrestrial planet maps from the literature in 
Section \ref{sec:discussion}. 
Finally, we conclude in Section \ref{sec:conclusion} with a look at future 
prospects for phase-curve mapping of terrestrial exoplanets.

\section{Data}
\label{sec:data}

\todo{(background extraction from 2-minute cadence targets)}

\todo{(outlier rejection method, trim down to x number of high-SNR lightcurves 
for computational tractability)}

\todo{(figure showing lightcurves for various targets)}

\section{Methods}
\label{sec:methods}

\subsection{Systematics Modeling}

While the astrophysical signal that we wish to analyze is present in all \todo{x} 
extracted lightcurves, its strength is modulated by time-variable systematics 
that are correlated but not identical across all sources. 
\todo{The observed lightcurves can be represented as... (systematics x 
Earthshine equation)}

We choose to model this systematic component using Pixel Level Decorrelation 
(PLD). \todo{(details, math, and a figure of the bases (possibly two-panel with 
comparison to telescope housekeeping variables?)}

\subsection{Reflected-Light Mapping with \starry}

\todo{(brief recap of \starry)}

\todo{(adapting \starry to reflected light/variable illumination phase - 
possibly make the gritty math details be an appendix?)}

\todo{(time variability of surface features)}

\subsection{Likelihood Function \& Inference}

\todo{(priors, regularization, equation for the likelihood)}

\todo{(details of TensorFlow optimization)}

\section{Results}
\label{sec:results}

\begin{figure}[t!]
    \begin{centering}
    \includegraphics[width=\linewidth]{figures/phases.pdf}
    \caption{\label{fig:phases}
             \tess view of Sol d on every date in Sectors 1
             and 2 that was included in the regression. Each image is
             labeled with the corresponding timestamp in TJD.
             \codelink{EarthView}
             }
    \end{centering}
\end{figure}

\todo{(present results with \& without time variability)}

\section{Discussion}
\label{sec:discussion}

\todo{(comparison of inferred features to ``model'' (Earth))}

\section{Conclusion}
\label{sec:conclusion}

\todo{(summarize)}

\todo{(utility for \TESS background removal)}

\todo{(talk abt prospects for doing this analysis with exoplanets)}


\acknowledgements{\todo{(Artist acknowledgement)} The authors gratefully 
acknowledge Dan Foreman-Mackey, David W Hogg, Ben Pope, \todo{and ...} 
for useful conversations. Crucial parts of this work were carried out at the 
TESS Data Workshop, hosted by Space Telescope Science Institute, and the 
Building Early Science with TESS Workshop, hosted by the University of Chicago.}
\facility{TESS}
\software{Astropy, Matplotlib, Numpy, \starry \citep{Luger2019}, TensorFlow}
% Bibliography
\pagebreak
\bibliography{bib}

\end{document}
