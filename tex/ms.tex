\documentclass[modern]{aastex62}

% Load the corTeX style definitions
% All the packages
\usepackage{url}
\usepackage{amsmath}
\usepackage{mathtools}
\usepackage{amssymb}
\usepackage{natbib}
\usepackage{graphicx}
\usepackage{calc}
\usepackage{etoolbox}
\usepackage{xspace}
\usepackage[T1]{fontenc} % https://tex.stackexchange.com/a/166791
\usepackage{textcomp}
\usepackage{ifxetex}
\ifxetex
\usepackage{fontspec}
\defaultfontfeatures{Extension = .otf}
\fi
\usepackage{fontawesome}
\usepackage{listings}

% Shorthand for this paper
\newcommand{\Python}{\textsf{Python}\xspace}
\newcommand{\cpp}{\textsf{C}++\xspace}
\newcommand{\starry}{\textsf{starry}\xspace}
\newcommand{\spiceypy}{\textsf{spiceypy}\xspace}
\newcommand{\tf}{\textsf{TensorFlow}\xspace}
\newcommand{\tess}{\emph{TESS}\xspace}

% References to text content
\newcommand{\documentname}{\textsl{article}}
\newcommand{\figureref}[1]{\ref{fig:#1}}
\newcommand{\Figure}[1]{Figure~\figureref{#1}}
\newcommand{\figurelabel}[1]{\label{fig:#1}}
\renewcommand{\eqref}[1]{\ref{eq:#1}}
\newcommand{\Eq}[1]{Equation~(\eqref{#1})}
\newcommand{\eq}[1]{\Eq{#1}}
\newcommand{\eqalt}[1]{Equation~\eqref{#1}}

% Add code, proof, and animation hyperlinks
\definecolor{linkcolor}{rgb}{0.1216,0.4667,0.7059}
\newcommand{\codeicon}{{\color{linkcolor}\faFileCodeO}}
\newcommand{\prooficon}{{\color{linkcolor}\faPencilSquareO}}
\newcommand{\codelink}[1]{\href{https://github.com/rodluger/earthshine/blob/3b1b4e288b46cb8823ff70e269c2acbf3303b236/notebooks/#1.ipynb}{\codeicon}\,\,}


% Define a proof environment for open source equation proofs
\newtagform{eqtag}[]{(}{)}
\newcommand{\currentlabel}{None}
\newenvironment{proof}[1]{%
\ifstrempty{#1}{%
\renewtagform{eqtag}[]{\raisebox{-0.1em}{{\color{red}\faPencilSquareO}}\,(}{)}%
}{%
\renewtagform{eqtag}[]{\prooflink{#1}\,(}{)}%
}%
\usetagform{eqtag}%
\renewcommand{\currentlabel}{#1}
\align%
}{%
\endalign%
\renewtagform{eqtag}[]{(}{)}%
\usetagform{eqtag}%
\message{<<<\currentlabel: \theequation>>>}%
}

% Define the `oscaption` command for open source figure captions
\newcommand{\oscaption}[2]{\caption{#2 \codelink{#1}}}

% Code examples
\definecolor{codegreen}{rgb}{0,0.6,0}
\definecolor{codegray}{rgb}{0.5,0.5,0.5}
\definecolor{codepurple}{rgb}{0.58,0,0.82}
\definecolor{backcolour}{rgb}{0.95,0.95,0.95}
\lstdefinestyle{mystyle}{
    backgroundcolor=\color{backcolour},
    commentstyle=\color{codegreen},
    keywordstyle=\color{magenta},
    numberstyle=\tiny\color{codegray},
    stringstyle=\color{codepurple},
    basicstyle=\small\ttfamily,
    breakatwhitespace=false,
    breaklines=true,
    captionpos=b,
    keepspaces=true,
    numbers=left,
    numbersep=5pt,
    showspaces=false,
    showstringspaces=false,
    showtabs=false,
    tabsize=2,
    aboveskip=1em,
    belowskip=1em,
    keywords=[2]{map},
    keywordstyle=[2]{\color{black!80!black}},
    upquote=true
}
\lstset{style=mystyle}

% Typography obsessions
\setlength{\parindent}{3.0ex}
\renewcommand\quad{\hskip\fontdimen3\font}


\usepackage{xcolor}
\usepackage{xspace}
\newcommand{\TESS}{\emph{TESS}\xspace}
\newcommand{\todo}[1]{\textcolor{red}{#1}}

% Bibliography stuff
\bibliographystyle{aasjournal}

% Begin!
\begin{document}

% Title
\title{Detection of Continents on a Habitable-Zone Terrestrial Planet with \TESS}

% Author list
\author[0000-0002-0296-3826]{Rodrigo Luger}
\email{rluger@flatironinstitute.org}
\affil{Center~for~Computational~Astrophysics, Flatiron~Institute, New~York, NY}
%
\author[0000-0002-9328-5652]{Megan Bedell}
\affil{Center~for~Computational~Astrophysics, Flatiron~Institute, New~York, NY}

\begin{abstract}
The Transiting Exoplanet Survey Satellite (\TESS) mission is a targeted effort to detect planets smaller than Neptune around bright, nearby stars \todo{(CITE)}. 
While \TESS is already enjoying great success with the discovery of many new worlds, the strongest signal in its data is typically ignored, as it lurks in the background of every camera pixel. 
In this work, we extract this signal and demonstrate that it corresponds to a terrestrial planet with a rotation period of 1 day \todo{(add error estimate!)}. 
Using the STARRY package, we are able to reconstruct the surface features of this rocky world \citep{Luger2018}. 
We find evidence for \todo{(clouds, some number of continents, whatever)}. 
We argue that this planet represents the most promising detection of a habitable world to date, although the potential intelligence of any life on it is yet to be determined.
\end{abstract}

\keywords{ALIENS}

% Introduction
\section{Introduction}
\label{sec:intro}
%

\section{Methods}

\section{Discussion}

\section{Conclusion}

% Bibliography
\pagebreak
\bibliography{bib}

\end{document}
